\documentclass[12pt]{article}
\usepackage{fullpage,hyperref}\setlength{\parskip}{3mm}\setlength{\parindent}{0mm}
\begin{document}

\begin{center}\bf
Homework 3. Due by 11:59pm on Sunday 9/21.

Academic misconduct

\end{center}
Read pages 15--23 of {\em On Being a Scientist} and Section 8.1 of the UM Rackham statement on academic misconduct at \url{https://rackham.umich.edu/academic-policies/section8/}. We are going to focus on plagiarism, a subtle topic that has gray areas. Write brief answers to the following questions, by editing the tex file available at \url{https://github.com/ionides/810f25}, and submit the resulting pdf file via Canvas. 

\begin{enumerate}

\item Over several years, I have found it is not unusual when reading STATS 810 homeworks (especially early in the semester) to find responses that include sentences matching the assigned reading word for word, without explicit attribution. Is this plagiarism?

Yes, I think it probably is. Because these responses are completely ideas from the assigned reading. When citing these ideas, explicit attribution should be made clear.

\item How do you think a GSI should respond when grading homework which they suspect contains an unattributed cut-and-paste contribution from ChatGPT or any other source?

"A faculty member or other reporting witness who finds evidence of academic misconduct must notify Rackham’s Resolution Officer and provide evidence in writing as soon as possible." (cited from UM Rackham statement on academic misconduct) Though I think it would be better to inform our instructor first and listen to his/her opinion. Not sure of that.

\item If you look, you will find common academic practices that are uncomfortably close to plagiarism, if this is strictly interpreted. For example,

(i) Homework problems may be copied or adapted from a textbook, without attribution.

(ii) Figures taken from papers and other internet sources may be presented in talks, class lectures, or GSI lab presentations, without attribution.

Should a responsible researcher attempt to avoid these RCRS gray areas? How?
  
Yes, a responsible researcher should try to avoid these. They should be more aware when they are trying to represent the words, ideas or any other materials from other people. Such citing behavior should always go with giving proper credit to the original source.

\item Are there any forms of inappropriate scientific conduct that you think have the combination of severity and prevalence to threaten the proper functioning of modern science? Are you more concerned about the total effect of serious (and presumably rare) misconduct, or milder (and potentially more common) misconduct?

Yes, the one that comes to my mind in the first place is falsification and fabrication. It might cause waste a lot of time and energy pf other researchers who are trying to use the result presented in this fabricated research. Or even worse, it may also cause severe trouble if the fabrication is not found early. I am more concerned with serious misconduct, because their effects are way worse than milder misconduct even though they are rare. I think it is like in real life, we often care more about crimes rather than some milder mistakes like over speeding.

\item Self-plagiarism is a subtle topic. When is it acceptable to copy/paste material you have already written into a draft you are currently working on? When is it inappropriate?

I think we should make it clear that such copy/paste is cited from materials that we wrote previously.
As long as the acknowledgement is clear, I think it should be acceptable. On the other hand, If we copy without explicit attribute, it is still inappropriate.

\item Suppose you use AI to help conduct your research, and for drafting or editing your research report. Can this amount to plagiarism? Usually, plagiarism is avoided by clearly attributing the source for each assertion in your writing---do you have advice for how to do this in practice when doing AI-assisted research?

YOUR ANSWER HERE.

\end{enumerate}
\end{document}
