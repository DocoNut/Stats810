\documentclass[12pt]{article}
\usepackage{fullpage,hyperref}\setlength{\parskip}{3mm}\setlength{\parindent}{0mm}
\begin{document}

\begin{center}\bf
Homework 5. Due by 11:59pm on Sunday 10/5.

Conflicts of interest and conflicts of commitment

\end{center}
Conflicts of interest (COI) and conflicts of commitment (COC) are a central topic for maintaining responsible scientific conduct since failure to manage COI/COC situations can lead to irresponsible conduct. Read pages 43--47 of {\em On Being a Scientist}. Write brief answers to the following questions, by editing the tex file available at \url{https://github.com/ionides/810f25}, and submit the resulting pdf file via Canvas.

\begin{enumerate}

\item What is the difference between a conflict of interest and a conflict of commitments? 

Conflict of interest refers to situations where personal interest might interfere professional judgements, while conflict of commitments refer to situations when responsibilities outside of research conflict with one's research.

\item Is there a clear delineation between these two ideas? If yes, explain why there is no ambiguity. If no, suggest a situation which might be hard to classify.

Yes, there is. Conflict of interest arises due to personal issues, while conflict of commitment arises due to management issues. I think they can exist for the same thing, but the reason of them would be different.

\item Give an example of a conflict of interest which might arise in an academic mentor/mentee relationship?

There is a conflict of interest when a mentor who wants to start a company require his/her mentees to work on company-related projects that are less related to their academic interests.

\item Give an example of a conflict of interest which might arise for an author of a published paper.

He could have published more narrowly focused papers that would build on his/her record of publication, but not help the field as quickly as that single paper.

\item You are asked to review a paper for a leading journal. You have high professional respect for the first author, and the paper looks interesting to you. You also count this author among your personal friends. Can you responsibly agree to review the paper? (Imagine you are giving advice to another friend who is in this situation.)

No. Personal relationships may create conflicts of interests. Some agencies even require researchers to identify these relationships.

\item Most PhD students have to balance time allocated to teaching (GSI) duties with their thesis research. Is this a conflict of interest, or a conflict of commitment, or both? What is your advice on how to manage this balance?

Conflict of commitment. There are limits of time spent for commitments outside research in many institutions. So my advice would be fulfill your responsibilities outside research but always prioritize research as long as you finish your job.

\item The two main ways to manage conflicts of interest are transparency and avoidance. Give an example of a conflict of interest best managed by avoidance and another best managed by transparency. Explain your answer.

Avoidance: For example, avoid researchers to bring projects outside of research to their students.

Transparency: The academic relationship should be public to reduce conflict of interest from personal relationship.

\end{enumerate}


\end{document}
