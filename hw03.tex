\documentclass[12pt]{article}
\usepackage{fullpage,hyperref}\setlength{\parskip}{3mm}\setlength{\parindent}{0mm}
\begin{document}

\begin{center}\bf
Homework 3. Due by 11:59pm on Sunday 9/21.

Academic misconduct

\end{center}
Read pages 15--23 of {\em On Being a Scientist} and Section 8.1 of the UM Rackham statement on academic misconduct at \url{https://rackham.umich.edu/academic-policies/section8/}. We are going to focus on plagiarism, a subtle topic that has gray areas. Write brief answers to the following questions, by editing the tex file available at \url{https://github.com/ionides/810f25}, and submit the resulting pdf file via Canvas. 

\begin{enumerate}

\item Over several years, I have found it is not unusual when reading STATS 810 homeworks (especially early in the semester) to find responses that include sentences matching the assigned reading word for word, without explicit attribution. Is this plagiarism?

YOUR ANSWER HERE.

\item How do you think a GSI should respond when grading homework which they suspect contains an unattributed cut-and-paste contribution from ChatGPT or any other source?

YOUR ANSWER HERE.

\item If you look, you will find common academic practices that are uncomfortably close to plagiarism, if this is strictly interpreted. For example,

(i) Homework problems may be copied or adapted from a textbook, without attribution.

(ii) Figures taken from papers and other internet sources may be presented in talks, class lectures, or GSI lab presentations, without attribution.

Should a responsible researcher attempt to avoid these RCRS gray areas? How?
  
YOUR ANSWER HERE.

\item Are there any forms of inappropriate scientific conduct that you think have the combination of severity and prevalence to threaten the proper functioning of modern science? Are you more concerned about the total effect of serious (and presumably rare) misconduct, or milder (and potentially more common) misconduct?

YOUR ANSWER HERE.

\item Self-plagiarism is a subtle topic. When is it acceptable to copy/paste material you have already written into a draft you are currently working on? When is it inappropriate?

YOUR ANSWER HERE.

\item Suppose you use AI to help conduct your research, and for drafting or editing your research report. Can this amount to plagiarism? Usually, plagiarism is avoided by clearly attributing the source for each assertion in your writing---do you have advice for how to do this in practice when doing AI-assisted research?

YOUR ANSWER HERE.

\end{enumerate}
\end{document}
