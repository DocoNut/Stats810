\documentclass[12pt]{article}
\usepackage{fullpage,hyperref}\setlength{\parskip}{3mm}\setlength{\parindent}{0mm}
\begin{document}

\begin{center}\bf
Homework 6. Due by 11:59pm on Sunday 10/19.

Collaborative research \& Human participants and animal subjects

\end{center}
Most statisticians do not have to worry directly about running large scientific research groups and dealing with the bureaucracy of ethical data collection. Yet, most statisticians collaborate sometimes with scientists who do. Read pages 24--28, 39--42 and 48--49 of {\em On Being a Scientist}.  Write brief answers to the following questions, by editing the tex file available at \url{https://github.com/ionides/810f25}, and submit the resulting pdf file via Canvas.

\begin{enumerate}

\item What is an IRB? Does a project studying aggregated observational data on human subjects (say, the total number of road accident injuries per state per year) need IRB approval to receive federal funding?

Institutional Review Boards. No, it does not, since the project just utilize data from human beings, but does not directly involve human participants. 

\item Suggest some ingredients which could lead to successful collaboration between two statisticians and/or between a statistician and a scientist.

Fairness, collegiality, and openness.

\item Collaborative group sizes can be small or large. Identify some strengths and weaknesses of larger collaborative groups relative to smaller collaborative groups.

Strengths: They can be experts in different fields, and help improve efficiency by division of labor; More inspirations and angles from larger teamwork.

Weakness: The management is more difficult; There could be more potential conflict of interest.
 
\item \label{p1} What are the advantages and disadvantages of being a conscientious collaborator who makes careful, thoughtful but timely contributions to the project, reads widely and takes the time to understand as much of the project as possible.

Advantages: This can help making the project better; he/she can also learn more from the project.

Disadvantages: This is inefficient. A better teamwork should divide the project into different parts, and distribute these parts according to different expertise of each team member, rather than let one person do much of the things.

\item Would you expect a PhD thesis adviser to act like the conscientious collaborator of question~\ref{p1} on your own thesis research? 

No, I want to be the main contributor to my own thesis. This way I learn more and earn the necessary credits.

\item What are some advantages and disadvantages of joining a project and then making a minimal contribution? Can this be responsible behavior? Consider the following example: you help a scientist carry out a statistical procedure and you help write up the paragraph describing it; you accept coauthorship on the resulting paper, without spending time on all other aspects of the paper.

Advantages: Time efficient for that person;
Disadvantage: For the team, this leads to discontent for other team members, and makes the process slower; For that person, he probably won't get his parts of credits.

\item How can one maintain a reasonable level of agreement within a collaboratoration on the expected involvement of each collaborator?

Just do your part, take responsibilities but do not other people's jobs.

\end{enumerate}
\end{document}
